\documentclass{article}
\usepackage{amsmath}
\usepackage{amssymb}
\usepackage{geometry}
\usepackage{graphicx}
\usepackage{booktabs}
\usepackage{fancyvrb}
\usepackage{bm}
\geometry{margin=1in}
\usepackage{float}


\begin{document}

\section*{Quasinewton Methods}

For all methods for all runs in this part, \textbf{stopping criterion:} $\|\mathbf{g}\| < 10^{-6}$ or max iter = 100

\section*{BFGS:}

% Run 1
\subsection*{Run 1: (scale = 0.5)}
\vspace{1em}

\textbf{Final iterate:}
\[
\bar{\mathbf{x}} =
\begin{bmatrix}
-1.0809 & -2.4388 & -0.9058 & -6.6190 & 4.5479 & 2.3531 \\
5.7014 & 4.0227 & 4.6624 & -3.3264 & -0.7440 & -1.4663
\end{bmatrix}
\]

\vspace{0.5em}
\noindent
\textbf{Distance to } $\mathbf{x}^*$: 13.175773

\vspace{0.5em}
\noindent
\boldsymbol{$\ell_k$} = 1.008635

\noindent
\boldsymbol{$q_k$} = 0.077213

\vspace{0.5em}
\noindent
\textbf{Runtime:} 3.2498 seconds

\begin{figure}[H]
    \centering
    \includegraphics[width=0.6\textwidth]{bfgs-1.png}
    \caption{Convergence behavior of Run 1 (scale = 0.5)}
    \label{fig:run1}
\end{figure}


% Run 2

\subsection*{Run 2: (scale = 0.75)}
\vspace{1em}

\textbf{Final iterate:}
\[
\bar{\mathbf{x}} =
\begin{bmatrix}
-0.9411 & -1.2555 & -0.6493 & 0.9411 & 1.2555 & 0.6493 \\
-0.5261 & -2.2526 & 0.4086 & 0.5261 & 2.2526 & -0.4086
\end{bmatrix}
\]

\vspace{0.5em}
\noindent
\textbf{Distance to } $\mathbf{x}^*$: 5.205394

\vspace{0.5em}
\noindent
\boldsymbol{$\ell_k$} = 1.000000

\noindent
\boldsymbol{$q_k$} = 0.192108

\vspace{0.5em}
\noindent
\textbf{Runtime:} 1.9939 seconds

\begin{figure}[H]
    \centering
    \includegraphics[width=0.6\textwidth]{bfgs-2.png}
    \caption{Convergence behavior of Run 2 (scale = 0.75)}
    \label{fig:run2}
\end{figure}


% Run 3

\subsection*{Run 3: (scale = 1.0)}
\vspace{1em}

\textbf{Final iterate:}
\[
\bar{\mathbf{x}} =
\begin{bmatrix}
-15.3170 & -3.1996 & 1.6094 & -16.3998 & -1.1521 & 1.5963 \\
-5.1207 & -19.8600 & 13.1906 & 27.2052 & -2.2500 & 2.2417
\end{bmatrix}
\]

\vspace{0.5em}
\noindent
\textbf{Distance to } $\mathbf{x}^*$: 43.367214

\vspace{0.5em}
\noindent
\boldsymbol{$\ell_k$} = 0.998854

\noindent
\boldsymbol{$q_k$} = 0.023006

\vspace{0.5em}
\noindent
\textbf{Runtime:} 3.1850 seconds

\begin{figure}[H]
    \centering
    \includegraphics[width=0.6\textwidth]{bfgs-3.png}
    \caption{Convergence behavior of Run 3 (scale = 1.0)}
    \label{fig:run3}
\end{figure}


% Run 4

\subsection*{Run 4: (scale = 1.5)}
\vspace{1em}

\textbf{Final iterate:}
\[
\bar{\mathbf{x}} =
\begin{bmatrix}
0.8902 & -5.0129 & -0.6499 & 8.7286 & 1.4997 & 1.7231 \\
-7.6670 & 1.4735 & -2.1226 & 0.7368 & -3.7933 & 0.6935
\end{bmatrix}
\]

\vspace{0.5em}
\noindent
\textbf{Distance to } $\mathbf{x}^*$: 13.790752

\vspace{0.5em}
\noindent
\boldsymbol{$\ell_k$} = 1.008206

\noindent
\boldsymbol{$q_k$} = 0.073707

\vspace{0.5em}
\noindent
\textbf{Runtime:} 3.1941 seconds

\begin{figure}[H]
    \centering
    \includegraphics[width=0.6\textwidth]{bfgs-4.png}
    \caption{Convergence behavior of Run 4 (scale = 1.5)}
    \label{fig:run4}
\end{figure}


% Run 5

\subsection*{Run 5: (scale = 2.0)}
\vspace{1em}

\textbf{Final iterate:}
\[
\bar{\mathbf{x}} =
\begin{bmatrix}
9.1294 & -4.5336 & 2.2928 & -10.5100 & -2.6251 & 4.2798 \\
10.7937 & 2.5347 & -3.4781 & -7.5699 & 4.3296 & 2.0809
\end{bmatrix}
\]

\vspace{0.5em}
\noindent
\textbf{Distance to } $\mathbf{x}^*$: 21.915950

\vspace{0.5em}
\noindent
\boldsymbol{$\ell_k$} = 1.011843

\noindent
\boldsymbol{$q_k$} = 0.046716

\vspace{0.5em}
\noindent
\textbf{Runtime:} 3.1972 seconds

\begin{figure}[H]
    \centering
    \includegraphics[width=0.6\textwidth]{bfgs-5.png}
    \caption{Convergence behavior of Run 5 (scale = 2.0)}
    \label{fig:run5}
\end{figure}

\vspace{0.5em}

\noindent
\textbf{Comments:} The BFGS method exhibited high variability in performance, with a strong dependence on the initial scaling. Run 2 was the most stable, converging to within a distance of 5.21 from $\mathbf{x}^*$, with $\ell_k = 1.000$ and $q_k = 0.192$. This suggests near-linear convergence, though not superlinear as expected from BFGS in ideal conditions.

It fits so well to sin(t) likely because it is biased as $\bar{\mathbf{x}}$ was estimated using BFGS for methods 3-5.

\section*{DFP:}

% Run 1
\subsection*{Run 1: (scale = 0.5)}
\vspace{1em}

\textbf{Final iterate:}
\[
\bar{\mathbf{x}} =
\begin{bmatrix}
-1.0781 & -1.5253 & 0.7567 & 0.5476 & 2.2466 & 0.4112 \\
1.4494 & 0.6951 & 0.0009 & 0.9409 & 1.2440 & -0.6370
\end{bmatrix}
\]

\vspace{0.5em}
\noindent
\textbf{Distance to } $\mathbf{x}^*$: 4.374717

\vspace{0.5em}
\noindent
\boldsymbol{$\ell_k$} = 1.000000

\noindent
\boldsymbol{$q_k$} = 0.228586

\vspace{0.5em}
\noindent
\textbf{Runtime:} 2.0782 seconds

\begin{figure}[H]
    \centering
    \includegraphics[width=0.6\textwidth]{dfp-1.png}
    \caption{Convergence behavior of Run 1 (scale = 0.5)}
    \label{fig:dfp-run1}
\end{figure}

% Run 2

\subsection*{Run 2: (scale = 0.75)}
\vspace{1em}

\textbf{Final iterate:}
\[
\bar{\mathbf{x}} =
\begin{bmatrix}
-1.1607 & 4.7025 & -0.8191 & -1.6038 & 3.0196 & 0.9323 \\
-1.0901 & -1.4623 & 0.7985 & 1.9986 & 2.5155 & -1.2523
\end{bmatrix}
\]

\vspace{0.5em}
\noindent
\textbf{Distance to } $\mathbf{x}^*$: 7.218019

\vspace{0.5em}
\noindent
\boldsymbol{$\ell_k$} = 1.000643

\noindent
\boldsymbol{$q_k$} = 0.138720

\vspace{0.5em}
\noindent
\textbf{Runtime:} 3.1931 seconds

\begin{figure}[H]
    \centering
    \includegraphics[width=0.6\textwidth]{dfp-2.png}
    \caption{Convergence behavior of Run 2 (scale = 0.75)}
    \label{fig:dfp-run2}
\end{figure}

% Run 3

\subsection*{Run 3: (scale = 1.0)}
\vspace{1em}

\textbf{Final iterate:}
\[
\bar{\mathbf{x}} =
\begin{bmatrix}
-1.0255 & -1.6297 & 0.7017 & -0.0642 & -0.1665 & 0.4478 \\
1.0808 & 1.4850 & -0.7929 & -0.1356 & -0.6414 & -0.7589
\end{bmatrix}
\]

\vspace{0.5em}
\noindent
\textbf{Distance to } $\mathbf{x}^*$: 4.093776

\vspace{0.5em}
\noindent
\boldsymbol{$\ell_k$} = 0.996106

\noindent
\boldsymbol{$q_k$} = 0.242375

\vspace{0.5em}
\noindent
\textbf{Runtime:} 0.5208 seconds

\begin{figure}[H]
    \centering
    \includegraphics[width=0.6\textwidth]{dfp-3.png}
    \caption{Convergence behavior of Run 3 (scale = 1.0)}
    \label{fig:dfp-run3}
\end{figure}

% Run 4

\subsection*{Run 4: (scale = 1.5)}
\vspace{1em}

\textbf{Final iterate:}
\[
\bar{\mathbf{x}} =
\begin{bmatrix}
0.4442 & 0.3916 & 0.0076 & -10.4362 & -0.3196 & 2.4098 \\
8.2042 & 0.7796 & -1.9122 & 4.4001 & -2.4851 & 2.4265
\end{bmatrix}
\]

\vspace{0.5em}
\noindent
\textbf{Distance to } $\mathbf{x}^*$: 14.920831

\vspace{0.5em}
\noindent
\boldsymbol{$\ell_k$} = 1.006463

\noindent
\boldsymbol{$q_k$} = 0.067890

\vspace{0.5em}
\noindent
\textbf{Runtime:} 3.2800 seconds

\begin{figure}[H]
    \centering
    \includegraphics[width=0.6\textwidth]{dfp-4.png}
    \caption{Convergence behavior of Run 4 (scale = 1.5)}
    \label{fig:dfp-run4}
\end{figure}

% Run 5

\subsection*{Run 5: (scale = 2.0)}
\vspace{1em}

\textbf{Final iterate:}
\[
\bar{\mathbf{x}} =
\begin{bmatrix}
-6.9806 & 1.6608 & -2.5250 & 0.2264 & 3.8056 & -1.1233 \\
1.0060 & 4.7691 & 30.0997 & 6.9964 & 1.5981 & -1.7032
\end{bmatrix}
\]

\vspace{0.5em}
\noindent
\textbf{Distance to } $\mathbf{x}^*$: 31.871335

\vspace{0.5em}
\noindent
\boldsymbol{$\ell_k$} = 1.003897

\noindent
\boldsymbol{$q_k$} = 0.031621

\vspace{0.5em}
\noindent
\textbf{Runtime:} 3.1991 seconds

\begin{figure}[H]
    \centering
    \includegraphics[width=0.6\textwidth]{dfp-5.png}
    \caption{Convergence behavior of Run 5 (scale = 2.0)}
    \label{fig:dfp-run5}
\end{figure}

\noindent
\textbf{Comments:} DFP displayed highly erratic performance across the tested scales. Runs 1 and 3 achieved relatively moderate distances to $\mathbf{x}^*$ (4.37 and 4.09, respectively), but exhibited $\ell_k$ values extremely close to 1, and $q_k$ values around 0.23–0.24, indicative of at best \textit{linear convergence}. In both cases, curvature updates were occasionally skipped due to near-zero inner products ($y^T s \approx 0$), undermining Hessian updates and contributing to stagnation.
However, the resulting $\varphi(\bar{\mathbf{x}}; t)$ in these runs still visually approximates $\sin(t)$ reasonably well, suggesting that even imperfect convergence can yield functionally useful results.




\section*{SR1 Results}

% Run 1
\subsection*{Run 1 (scale = 0.5)}
\vspace{1em}

\textbf{Final iterate:}
\[
\bar{\mathbf{x}} =
\begin{bmatrix}
1.0918 & 1.5568 & 0.7157 & -1.0687 & -1.6807 & 0.6749 \\
-0.3680 & -0.6154 & -0.2197 & -0.0035 & -2.3843 & 0.6564
\end{bmatrix}
\]

\vspace{0.5em}
\noindent
\textbf{Distance to } $\mathbf{x}^*$: 4.091887

\vspace{0.5em}
\noindent
$\ell_k = 1.002692$

\noindent
$q_k = 0.245704$

\vspace{0.5em}
\noindent
\textbf{Runtime:} 2.0340 seconds

\begin{figure}[H]
    \centering
    \includegraphics[width=0.6\textwidth]{sr1-1.png}
    \caption{Convergence behavior of Run 1 (scale = 0.5)}
\end{figure}

% Run 5

\subsection*{Run 2 (scale = 0.75)}
\vspace{1em}

\textbf{Final iterate:}
\[
\bar{\mathbf{x}} =
\begin{bmatrix}
-0.7811 & -0.9754 & -0.6688 & 0.9402 & 1.7955 & 0.6204 \\
0.4541 & 0.6732 & -0.6821 & -0.7157 & -2.0541 & -0.5081
\end{bmatrix}
\]

\vspace{0.5em}
\noindent
\textbf{Distance to } $\mathbf{x}^*$: 4.514975

\vspace{0.5em}
\noindent
$\ell_k = 1.026155$

\noindent
$q_k = 0.233223$

\vspace{0.5em}
\noindent
\textbf{Runtime:} 2.0378 seconds

\begin{figure}[H]
    \centering
    \includegraphics[width=0.6\textwidth]{sr1-2.png}
    \caption{Convergence behavior of Run 2 (scale = 0.75)}
\end{figure}

% Run 5

\subsection*{Run 3 (scale = 1.0)}
\vspace{1em}

\textbf{Final iterate:}
\[
\bar{\mathbf{x}} =
\begin{bmatrix}
0.5261 & 2.2526 & -0.4086 & -0.5261 & -2.2526 & -0.4085 \\
0.9411 & 1.2555 & -0.6493 & -0.9411 & -1.2555 & -0.6493
\end{bmatrix}
\]

\vspace{0.5em}
\noindent
\textbf{Distance to } $\mathbf{x}^*$: 4.457463

\vspace{0.5em}
\noindent
$\ell_k = 1.000000$

\noindent
$q_k = 0.224343$

\vspace{0.5em}
\noindent
\textbf{Runtime:} 2.0291 seconds

\begin{figure}[H]
    \centering
    \includegraphics[width=0.6\textwidth]{sr1-3.png}
    \caption{Convergence behavior of Run 3 (scale = 1.0)}
\end{figure}

% Run 5

\subsection*{Run 4 (scale = 1.5)}
\vspace{1em}

\textbf{Final iterate:}
\[
\bar{\mathbf{x}} =
\begin{bmatrix}
0.7247 & 0.9862 & 0.0200 & -0.4729 & -6.7756 & 0.5235 \\
-3.6530 & -1.7786 & 1.3066 & 2.4309 & -1.9672 & 2.2817
\end{bmatrix}
\]

\vspace{0.5em}
\noindent
\textbf{Distance to } $\mathbf{x}^*$: 8.936805

\vspace{0.5em}
\noindent
$\ell_k = 0.942776$

\noindent
$q_k = 0.099457$

\vspace{0.5em}
\noindent
\textbf{Runtime:} 2.0902 seconds

\begin{figure}[H]
    \centering
    \includegraphics[width=0.6\textwidth]{sr1-4.png}
    \caption{Convergence behavior of Run 4 (scale = 1.5)}
\end{figure}



\subsection*{Run 5 (scale = 2.0)}
\vspace{1em}

\textbf{Stopping criterion:} $\|g\| < 1 \times 10^{-6}$ or max\_iter = 100

\vspace{0.5em}
\textbf{Final iterate:}
\[
\bar{\mathbf{x}} =
\begin{bmatrix}
0.7866 & -0.7551 & -0.0173 & -1.4401 & -1.5910 & 0.8463 \\
0.0306 & -2.6672 & 8.6018 & 0.5695 & -22.2815 & 18.4644
\end{bmatrix}
\]

\vspace{0.5em}
\noindent
\textbf{Distance to } $\mathbf{x}^*$: 29.854298

\vspace{0.5em}
\noindent
$\ell_k = 1.034099$

\noindent
$q_k = 0.035819$

\vspace{0.5em}
\noindent
\textbf{Runtime:} 2.0566 seconds

\begin{figure}[H]
    \centering
    \includegraphics[width=0.6\textwidth]{sr1-5.png}
    \caption{Convergence behavior of Run 5 (scale = 2.0)}
\end{figure}

\noindent
\textbf{Comments:} SR1 generally performed with more stability than DFP, although it too showed signs of divergence at higher scales. Runs 1 through 3 consistently yielded distances in the range of 4.09 to 4.51, with $\ell_k \approx 1.00$ and $q_k$ around 0.22–0.25. These values imply \textit{steady but non-accelerating convergence}, suggestive of near-linear behavior. However, despite this steady convergence, SR1's final approximations of $\sin(t)$ were less accurate compared to those achieved by DFP and BFGS.

\end{document}
